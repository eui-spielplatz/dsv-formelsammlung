\documentclass[10pt,a4paper]{article}
\usepackage[utf8]{inputenc}
\usepackage{longtable}
\usepackage{titlesec}
\usepackage{amsfonts}
\usepackage{graphicx}
\usepackage{hyperref}
\usepackage{amsmath}
\usepackage{amssymb}
\usepackage{multicol}
\usepackage{multirow}
\usepackage{array}
\usepackage{color}
\usepackage[dvipsnames]{xcolor}
\usepackage{adjustbox}
\usepackage{float}
\usepackage[left=2cm, right=2.60cm, top=2cm, bottom=2.50cm]{geometry}
\usepackage[txtcentered=true, height=40pt, width=70pt]{thumbs}
\usepackage[german]{babel}
\usepackage[linewidth=1pt]{mdframed}
\usepackage{subcaption}
\usepackage{tikz-cd}
\usepackage[justification=centering]{caption}

\newcommand{\fancythumb}[2]{
	\addthumb{#1}{\large\sffamily\textbf{\space\space#1\vspace{5pt}}}{white}{#2}
}

\newcommand{\fancyformula}[2]{
	\small\raggedright{\sffamily\textbf{#1}}
	#2
}

\newcommand{\ftransform}{
	~\xrightarrow{~\mathcal{F}~}~
}
\newcommand{\ztransform}{
	~\xrightarrow{~\mathcal{Z}~}~
}
\newcommand{\invztransform}{
	~\xrightarrow{~\mathcal{Z}^{-1}~}~
}
\newcommand{\ztransformOneSided}{
	~\xrightarrow{~\mathcal{Z^{+}}~}~
}

\DeclareMathOperator{\sinc}{sinc}
\DeclareMathOperator{\sgn}{sgn}
\DeclareMathOperator{\rect}{rect}
\DeclareMathOperator{\tri}{tri}
\DeclareMathOperator{\acos}{acos}
\DeclareMathOperator{\asin}{asin}
\renewcommand{\Re}{\operatorname{Re}}
\renewcommand{\Im}{\operatorname{Im}}

\pagenumbering{arabic}
\titleformat*{\section}{\sffamily\Large\bfseries}
\titleformat*{\subsection}{\sffamily\large\bfseries}
\titleformat*{\subsubsection}{\sffamily\normalsize\bfseries}

\begin{document}
\section*{Zweiseitige $z$-Transformation}
\subsection*{Definition}
\vspace{-1.5em}
\begin{multicols}{2}
	\noindent
	\[ x(n)\ztransform X(z)=\sum_{n=-\infty}^{\infty}x(n)\cdot z^{-n} \]
	\[x(n)=\frac{1}{2\pi j}\oint_{C}X(z)\cdot z^{n-1}dz \]
\end{multicols}

\subsection*{Eigenschaften}
\vspace{-1.5em}
\begin{multicols}{2}
	\fancyformula{Spiegelung}{
		\[ x(-n) \ztransform X \left(z^{-1} \right) \]
		\footnotesize Wenn für $X(z)$ gilt $\mathrm{KB} = \{ z ~ | ~ r < |z| < r_2 \}$, dann ist nun $\mathrm{KB} = \left \{ z ~ | ~ \frac{1}{r_2} < |z| < \frac{1}{r_1} \right \}$.
		\vspace{0.5em}
	}

	\fancyformula{Konjugiert komplex}{
		\[ x^{*}(n)\ztransform X^{*}(z^{*}) \]
		\footnotesize Keine Änderung des KB.
		\vspace{0.5em}
	}

	\fancyformula{Linearität}{
		\[ \sum_i c_i ~ x_i(n) \ztransform \sum_i c_i ~ X_i(z) \]
		\footnotesize Mindestens $\mathrm{KB} = \bigcap_i \mathrm{KB}_i$.
		\vspace{0.5em}
	}

	\fancyformula{Verschiebung}{
		\[ x(n-k) \ztransform z^{-k} ~ X(z) \]
		\footnotesize Bleibt gleich, bis auf: $z = 0$ fällt weg für $k > 0$ und $z = \infty$ fällt weg für $k < 0$.
		\vspace{0.5em}
	}

	\fancyformula{Dehnung in $z$-Ebene}{
		\[ a^n ~ x(n) \ztransform X\left(\frac{z}{a}\right) \quad \text{mit} ~ a \neq 0 \]
		\footnotesize Wenn für $X(z)$ gilt $\mathrm{KB} = \{ z ~ | ~ r < |z| < r_2 \}$, dann ist nun $\mathrm{KB} = \left \{ z ~ | ~ |a| ~ r_1 < |z| < |a| ~ r_2 \right \}$.
		\vspace{0.5em}
	}

	\fancyformula{Differentiation}{
		\[ n ~ x(n) \ztransform -z ~ \frac{\mathrm d}{\mathrm dz} ~ X(z) \]
		\footnotesize Keine Änderung des KB, sofern $X(z)$ rational.
		\vspace{0.5em}
	}

	\fancyformula{Faltung}{
		\[ (x_1 * x_2)(n) \ztransform X_{1}(z) ~ X_{2}(z)\]
		\footnotesize Mindestens $\mathrm{KB} = \mathrm{KB}_1 \cap \mathrm{KB}_2$.
		\vspace{0.5em}
	}

	\fancyformula{Moment}{
		\[ \sum_{n=-\infty}^{\infty} n ~ x(n) = -z ~ \frac{\mathrm d}{\mathrm dz} ~ X(z) \bigg|_{z=1} \]
	}

	\fancyformula{Anfangswert}{
		\[
			x(0) = \begin{cases}
				X(\infty) & \text{wenn $x(n)$ rechtsseitig ist} \\
				X(0) & \text{wenn $x(n)$ linksseitig ist}
			\end{cases}
		\]
	}

	{\small\raggedright\sffamily\textbf{Eindeutigkeit}}
	\[
		\begin{tikzcd}[every arrow/.append style={shift left}, column sep = 9em]
			x(n) \arrow{r}{\text{eindeutig}} & X(z) \arrow{l}{\text{nur mit\newline KB eindeutig}}
		\end{tikzcd}
	\]

	\fancyformula{Fouriertransformation}{
		\[ X(e^{j\omega}) := \sum_{n = -\infty}^{\infty} x(n) ~ e^{j\omega n} \overset{!}{=} X(z = e^{j\omega}) \]
	}
\end{multicols}

\subsection*{Wichtige zweiseitige $z$-Transformationen}
\vspace{-1.5em}
\begin{center}
	\setlength\extrarowheight{8pt}
	\begin{longtable}{r c l l}
		$x(n)$ & $\ztransform$ & $ X(z)$ & Konvergenzbereich KB \\ \hline
		$\delta(n)$ & $\ztransform$ & $1$ & $\mathbb{C}$ \\
		$\delta(n-k)$ & $\ztransform$ & $z^{-k}$ & $\mathbb C \setminus \{0\} ~ \text{wenn} ~ k > 0$ \\
		&&&$\mathbb C \setminus \{\infty\} ~ \text{wenn} ~ k < 0$ \\
		$a^n ~ u(n)$ & $\ztransform$ & $\dfrac{1}{1 - az^{-1}}$ & $|z| > |a|$ \\
		$-a^n ~ u(-n-1)$ & $\ztransform$ & $\dfrac{1}{1 - az^{-1}}$ & $|z| < |a|$ \\
		$n ~ a^n ~ u(n)$ & $\ztransform$ & $\dfrac{az^{-1}}{(1 - az^{-1})^2}$ & $|z| > |a|$ \\
		$-n ~ a^n ~ u(-n-1)$ & $\ztransform$ & $\dfrac{az^{-1}}{(1 - az^{-1})^2}$ & $|z| < |a|$ \\
		$d_k(n) ~ a^n ~ u(n)$ & $\ztransform$ & $\dfrac{1}{(1-az^{-1})^k}$ & $|z| > |a|$ \\
		$-d_k(n) ~ a^{n} ~ u(-n-k)$ & $\ztransform$ & $\dfrac{1}{(1-az^{-1})^k}$ & $|z| < |a|$ \\
		$a^{|n|}$ & $\ztransform$ & $\dfrac{1-a^2}{(1-az^{-1})(1-az)}$ & $|a|<|z|<\frac{1}{|a|}$ \\
		$a^n \cos(\omega_{0}n) ~ u(n)$ & $\ztransform$ & $\dfrac{1 - a ~ \cos(\omega_0) ~ z^{-1}}{1-2 a ~ \cos(\omega_{0}) ~ z^{-1} + a^2 ~ z^{-2}}$ & $|z| > |a|$ \\
		$a^n ~ \sin(\omega_0 n) ~ u(n)$ & $\ztransform$ & $\dfrac{a ~ \sin(\omega_{0}) ~ z^{-1}}{1 - 2 a ~ \cos(\omega_{0}) ~ z^{-1} + a^2 ~ z^{-2}}$ & $|z| > |a|$ \\
		$\dfrac{1}{n!} ~ u(n)$ & $\ztransform$ & $e^{z^{-1}}$ & $\mathbb C \setminus \{0\}$
	\end{longtable}
\end{center}

Mit Abkürzung: $d_{1}(n)=1$, $d_{2}(n)=n+1$, $d_{3}(n)=\frac{(n+1)(n+2)}{2}$, d$_{k}(n)=\sum_{i=0}^{n}d_{k-1}(i)$

\subsection*{Rationale $z$-Transformierte}
\vspace{-1.5em}
\begin{multicols}{2}
	\fancyformula{Definition}{
		\[ X(z) = \dfrac{B(z)}{A(z)} = \dfrac{b_0 + b_1 z^{-1} + \cdots + b_M z^{-M}}{a_0 + a_1 z^{-1} + \cdots + a_N z^{-N}} \]
	}

	\fancyformula{Polstellen $p_i$ und Nullstellen $n_i$}{
		\[ X(z = p_i) = \infty ~ \Leftrightarrow ~ B(p_i) = \infty ~ \text{oder} ~ A(p_i) = 0 \]
		\[ X(z = n_i) = 0 ~ \Leftrightarrow ~ B(n_i) = 0 ~ \text{oder} ~ A(n_i) = \infty \]
	}

	\fancyformula{Eigenschaften}{
		\begin{itemize}
			\item Keine Polstellen im KB von $X(z)$
			\item Ein Pol-Nullstellen-Diagramm legt $X(z)$ bis auf eine \textbf{Skalierung} fest
		\end{itemize}
	}

	\fancyformula{Pol-Nullstellen-Zerlegung}{
		\[
			X(z) = \frac{b_{0}}{a_{0}} ~ z^{N-M} ~ \frac{(z-n_{1})\cdot \ldots \cdot(z-n_{M})}{(z-p_{1})\cdot \ldots \cdot(z-p_{N})}
		\]

		\begin{align*}
			\Rightarrow & \begin{cases}
				\text{M nicht triviale Nullstellen } n_{1},...,n_{M}\\
				\text{N nicht triviale Polstellen } p_{1},...,p_{N}
			\end{cases}\\
			+ & \begin{cases}
				N > M\text{: $N - M$ triviale Nullstellen an $z = 0$}\\
				N < M \text{: $M - N$ triviale Polstellen an $z = 0$}
			\end{cases}
		\end{align*}
	}

	{\small\raggedright\sffamily\textbf{Pol-Nullstellen-Kürzung für $p_i = n_j$}}

	\begin{itemize}
		\item Reduzierte Systemordnung
		\item Mögliche Änderung von Stabilität und KB
	\end{itemize}


	% TODO: Zu viel Text, zu lang
	%{\small\raggedright\sffamily\textbf{Zeitverhalten von Polstellen}}
	%
	%Wenn $x(n)$ \textcolor{teal}{rechtsseitig} / \textcolor{red}{linksseitig} ist und wenn:
	%\begin{itemize}
	%	\item Polstellen \textcolor{teal}{innerhalb} / \textcolor{red}{außerhalb} des Einheitskreises liegen, dann klingt $|x(n)|$ ab
	%	\item Polstellen \textcolor{teal}{außerhalb} / \textcolor{red}{innerhalb} des Einheitskreises liegen, dann wächst $|x(n)|$ unbeschränkt
	%	\item Einfache Polstellen auf dem Einheitskreis liegen, dann bleibt $|x(n)|$ beschränkt
	%	\item Mehrfache Polstellen auf dem Einheitskreis liegen, dann wächst $|x(n)|$ unbeschränkt
	%	\item Je kleiner der Abstand von Polstellen zum Einheitskreis ist, desto langsamer das Abklingen/Wachsen von  $|x(n)|$
	%	\item Je größer der Abstand von Polstellen zum Einheitskreis ist, desto schneller das Abklingen/Wachsen von $|x(n)|$
	%\end{itemize}
\end{multicols}

\subsection*{Rücktransformation $X(z) \invztransform x(n)$}
\subsubsection*{Möglichkeit 1: Ablesen mit Potenzreihenenwicklung}
Bringe durch Ausmultiplizieren oder Potenzreihenenwicklung auf Form:
\[
	X(z) \overset{!}{=} \sum_n x(n) ~ z^{-n} ~ \Rightarrow ~ \text{ dann direktes Ablesen von $x(n)$}
\]

\subsubsection*{Möglichkeit 2: Partialbruchzerlegung für rationale $z$-Transformierte}
\[
	\text{Sei} ~ X(z) = \frac{\beta(z^{-1})}{\alpha(z^{-1})} ~ \text{rationale $z$-Transformierte}
\]

\begin{enumerate}
\item \textbf{Polynomdivision} falls Zählergrad $\geq$ Nennergrad!
\item \textbf{Partialbruchzerlegung} und Bestimmung der Koeffizienten wie herkömmlich oder mit Regeln:
\begin{figure}[H]
	\centering
	\setlength{\fboxsep}{10pt}
	\begin{adjustbox}{valign=h,center=0.42\textwidth,margin=1ex,minipage=[t][][t]{0.37\textwidth},bgcolor=white,cfbox=black 1.5pt 0pt 1pt}
		Koeffizient $c_i$ für \textbf{einfache Polstelle} $p_i$
		\[ \frac{\beta(z^{-1})}{\alpha(z^{-1})} = \frac{c_1}{1 - p_1 z^{-1}} + \cdots + \frac{c_N}{1 - p_N z^{-1}} \]
		\[ c_i = \frac{\beta(z^{-1})}{\alpha(z^{-1})} ~ (1 - p_i z^{-1}) \bigg |_{z = p_{i}} \]
		\vspace{3.8em}
	\end{adjustbox}
	\begin{adjustbox}{valign=h,center=0.48\textwidth,margin=1ex,minipage=[t][][t]{0.43\textwidth},bgcolor=white,cfbox=black 1.5pt 0pt 1pt}
		Koeffizienten $c_1, \ldots, c_m$ für \textbf{m-fache Polstellen} $p_1, \ldots, p_m$
		\[ \frac{\beta(z^{-1})}{\alpha(z^{-1})}=\frac{c_{1}}{1-p_{1}z^{-1}}+\cdots+\frac{c_{m}}{(1-p_{1}z^{-1})^m} + \ldots \]
		\[ c_{m - i} = \frac{1}{i! ~ (-p_1)^{i}}\frac{\mathrm d^{i} \gamma(w)}{\mathrm d w^{i}} \bigg |_{w = p_1^{-1}} \text{mit} ~ 0 \leq i < m \]
		\[ \gamma(w) := \frac{\beta(z^{-1})}{\alpha(z^{-1})} ~ \left(1 - p_1 z^{-1} \right)^m \bigg |_{z^{-1} = w} \]
	\end{adjustbox}
\end{figure}

\item \textbf{Rücktransformation} mittels Tabelle bekannter $z$-Transformationen

\end{enumerate}

\subsection*{LTI-Systeme und $z$-Transformation}
\subsubsection*{Übertragungsfunktion}
\[
	y(n) = (h * x)(n) \ztransform Y(z) = H(z) ~ X(z)
\]
\[
	H(z) = \frac{Y(z)}{X(z)} = \frac{\beta(z^{-1})}{\alpha(z^{-1})}
\]
	Differenzengleichung im Zeitbereich $\leftrightarrow H(z)$ rational mit Pol-und Nullstellen\\

\subsubsection*{BIBO-Stabilität}
\begin{align*}
	& \text{System ist BIBO-stabil} \\
	\Leftrightarrow \quad & \textstyle \sum_{n=-\infty}^\infty |h(n)| < \infty
	& \Leftrightarrow \quad & \text{$h(n)$ ist kausal und $|p_i| < 1 ~ \forall i$} \\
	\Leftrightarrow \quad & \text{KB von $H(z)$ enthält den Einheitskreis $|z| = 1$}
	& \Leftrightarrow \quad & \text{$h(n)$ ist antikausal und $|p_i| > 1 ~ \forall i$}
\end{align*}

\section*{einseitige z-Transformation}
\subsection*{Definition}
\begin{multicols}{2}
	\fancyformula{Hintransformation}{
	\[x(n) \ztransformOneSided \sum_{n=\textcolor{red}{0}}^{\infty}x(n)z^{-n}\]
	}
	\fancyformula{Rücktransformation}{
	\[x(n)=\frac{1}{2\pi j}\oint_{C}X^{+}(z)z^{n-1}dz \text{ mit }n \geq 0\]
	}
\end{multicols}
\subsection*{Eigenschaften}
\begin{multicols}{2}
	 \fancyformula{Spiegelung}{
 	\[x(-n) \ztransformOneSided \text{nicht definiert}\]
	}
	\fancyformula{konjugiert komplex}{
	\[x^{*}(n) \ztransformOneSided X^{+*}(z^{*})\]
	}
	\fancyformula{Linearität}{
	\[\sum_{i}^{}c_{i}x_{i}(n)\ztransformOneSided \sum_{i}^{}c_{i}X^{+}_{i}(z)\]
	}
	\fancyformula{Verschiebung mit $k>0$}{
	\[x(n-k)\ztransformOneSided z^{-k}\left[X^{+}(z)+\textcolor{red}{\sum_{n=-k}^{-1}x(n)z^{-n}}\right]\]
	\[x(n+k)\ztransformOneSided z^{k}\left[X^{+}(z)-\textcolor{red}{\sum_{n=0}^{k-1}x(n)z^{-n}}\right]\]
	}
	\fancyformula{Dehnung}{
	\[a^{n}x(n)\ztransformOneSided X^{+}\left(\frac{z}{a}\right) \text{mit }a\neq 0\]
	}
	\fancyformula{Differentiation}{
	\[nx(n)\ztransformOneSided -z\frac{d}{dz}X^{+}(z)\]
	}
	\fancyformula{Moment}{
	\[\sum_{n=0}^{\infty}nx(n)=-z\frac{d}{dz}X^{+}(z)|_{z=1}\]
	}
	\fancyformula{Faltung}{
	\[(x_{1}*x_{2})(n)\ztransformOneSided X_{1}^{+}(n)X_{2}^{+}(n) \text{ mit $x_{i}(n)$ rechtsseitig}\]
	}
	\fancyformula{Anfangswert}{
	\[x(0)=X(\infty)\]
	}
\end{multicols}
\subsection*{Lösen von Differenzengleichungen}
\begin{enumerate}
	\item einseitige z-Transformation der Differenzengleichung mit Anfangsbedingungen
	\item nach $Y^{+}(z)$ umformen
	\item durch Rücktransformation $y^{+}(n)$ z.B. mit  Partialbruchzerlegung bestimmen
	\item  $y(n)$ aus den Anfangsbedingungen und $y^{+}(n)$  bestimmen
\end{enumerate}
\end{document}
