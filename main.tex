\documentclass[10pt,a4paper]{article}
\usepackage[utf8]{inputenc}
\usepackage{longtable}
\usepackage{titlesec}
\usepackage{amsfonts}
\usepackage{graphicx}
\usepackage{hyperref}
\usepackage{amsmath}
\usepackage{amssymb}
\usepackage{multicol}
\usepackage{multirow}
\usepackage{array}
\usepackage{color}
\usepackage[dvipsnames]{xcolor}
\usepackage{adjustbox}
\usepackage{float}
\usepackage[left=2cm, right=2.60cm, top=2cm, bottom=2.50cm]{geometry}
\usepackage[txtcentered=true, height=40pt, width=70pt]{thumbs}
\usepackage[german]{babel}
\usepackage[linewidth=1pt]{mdframed}
\usepackage{subcaption}
\usepackage{pst-sigsys,pst-plot,pstricks-add}
\usepackage{tikz-cd}
\usepackage[justification=centering]{caption}

\newcommand{\fancythumb}[2]{
	\addthumb{#1}{\large\sffamily\textbf{\space\space#1\vspace{5pt}}}{white}{#2}
}

\newcommand{\fancyformula}[2]{
	\small\raggedright{\sffamily\textbf{#1}}
	#2
}

\newcommand{\ftransform}{
	~\xrightarrow{~\mathcal{F}~}~
}
\newcommand{\ztransform}{
	~\xrightarrow{~\mathcal{Z}~}~
}
\newcommand{\invztransform}{
	~\xrightarrow{~\mathcal{Z}^{-1}~}~
}
\newcommand{\ztransformOneSided}{
	~\xrightarrow{~\mathcal{Z^{+}}~}~
}

\DeclareMathOperator{\sinc}{sinc}
\DeclareMathOperator{\sgn}{sgn}
\DeclareMathOperator{\rect}{rect}
\DeclareMathOperator{\tri}{tri}
\DeclareMathOperator{\acos}{acos}
\DeclareMathOperator{\asin}{asin}
\renewcommand{\Re}{\operatorname{Re}}
\renewcommand{\Im}{\operatorname{Im}}

\pagenumbering{arabic}
\titleformat*{\section}{\sffamily\Large\bfseries}
\titleformat*{\subsection}{\sffamily\large\bfseries}
\titleformat*{\subsubsection}{\sffamily\normalsize\bfseries}

\begin{document}
\section*{Zweiseitige $z$-Transformation}
\subsection*{Definition}
\vspace{-1.5em}
\begin{multicols}{2}
	\noindent
	\[ x(n)\ztransform X(z) = \sum_{n=-\infty}^{\infty} x(n) ~ z^{-n} \]
	\[ x(n) = \frac{1}{2\pi j} \oint_C X(z) ~ z^{n-1} ~ \mathrm dz \]
\end{multicols}

\subsection*{Eigenschaften}
\vspace{-1.5em}
\begin{multicols}{2}
	\fancyformula{Spiegelung}{
		\[ x(-n) \ztransform X \left(z^{-1} \right) \]
		\footnotesize Wenn für $X(z)$ gilt $\mathrm{KB} = \{ z ~ | ~ r < |z| < r_2 \}$, dann ist nun $\mathrm{KB} = \left \{ z ~ | ~ \frac{1}{r_2} < |z| < \frac{1}{r_1} \right \}$.
		\vspace{0.5em}
	}

	\fancyformula{Konjugiert komplex}{
		\[ x^{*}(n)\ztransform X^{*}(z^{*}) \]
		\footnotesize Keine Änderung des KB.
		\vspace{0.5em}
	}

	\fancyformula{Linearität}{
		\[ \sum_i c_i ~ x_i(n) \ztransform \sum_i c_i ~ X_i(z) \]
		\footnotesize Mindestens $\mathrm{KB} = \bigcap_i \mathrm{KB}_i$.
		\vspace{0.5em}
	}

	\fancyformula{Verschiebung}{
		\[ x(n-k) \ztransform z^{-k} ~ X(z) \]
		\footnotesize Bleibt gleich, bis auf: $z = 0$ fällt weg für $k > 0$ und $z = \infty$ fällt weg für $k < 0$.
		\vspace{0.5em}
	}

	\fancyformula{Dehnung in $z$-Ebene}{
		\[ a^n ~ x(n) \ztransform X\left(\frac{z}{a}\right) \quad \text{mit} ~ a \neq 0 \]
		\footnotesize Wenn für $X(z)$ gilt $\mathrm{KB} = \{ z ~ | ~ r < |z| < r_2 \}$, dann ist nun $\mathrm{KB} = \left \{ z ~ | ~ |a| ~ r_1 < |z| < |a| ~ r_2 \right \}$.
		\vspace{0.5em}
	}

	\fancyformula{Differentiation in $z$-Ebene}{
		\[ n ~ x(n) \ztransform -z ~ \frac{\mathrm d}{\mathrm dz} ~ X(z) \]
		\footnotesize Keine Änderung des KB, sofern $X(z)$ rational.
		\vspace{0.5em}
	}

	\fancyformula{Faltung}{
		\[ (x_1 * x_2)(n) \ztransform X_{1}(z) ~ X_{2}(z)\]
		\footnotesize Mindestens $\mathrm{KB} = \mathrm{KB}_1 \cap \mathrm{KB}_2$.
		\vspace{0.5em}
	}

	\fancyformula{Moment}{
		\[ \sum_{n=-\infty}^{\infty} n ~ x(n) = -z ~ \frac{\mathrm d}{\mathrm dz} ~ X(z) \bigg|_{z=1} \]
	}

	\fancyformula{Anfangswert}{
		\[
			x(0) = \begin{cases}
				X(\infty) & \text{wenn $x(n)$ rechtsseitig ist} \\
				X(0) & \text{wenn $x(n)$ linksseitig ist}
			\end{cases}
		\]
	}

	{\small\raggedright\sffamily\textbf{Eindeutigkeit}}
	\[
		\begin{tikzcd}[every arrow/.append style={shift left}, column sep = 9em]
			x(n) \arrow{r}{\text{eindeutig}} & X(z) \arrow{l}{\text{nur mit\newline KB eindeutig}}
		\end{tikzcd}
	\]

	\fancyformula{Fouriertransformation}{
		\[ X(e^{j\omega}) := \sum_{n = -\infty}^{\infty} x(n) ~ e^{j\omega n} \overset{!}{=} X(z = e^{j\omega}) \]
	}
\end{multicols}

\subsection*{Wichtige zweiseitige $z$-Transformationen}
\vspace{-1.5em}
\begin{center}
	\setlength\extrarowheight{8pt}
	\begin{longtable}{r c l l}
		$x(n)$ & $\ztransform$ & $ X(z)$ & Konvergenzbereich KB \\ \hline
		$\delta(n)$ & $\ztransform$ & $1$ & $\mathbb{C}$ \\
		$\delta(n-k)$ & $\ztransform$ & $z^{-k}$ & $\mathbb C \setminus \{0\} ~ \text{wenn} ~ k > 0$ \\
		&&&$\mathbb C \setminus \{\infty\} ~ \text{wenn} ~ k < 0$ \\
		$a^n ~ u(n)$ & $\ztransform$ & $\dfrac{1}{1 - az^{-1}}$ & $|z| > |a|$ \\
		$-a^n ~ u(-n-1)$ & $\ztransform$ & $\dfrac{1}{1 - az^{-1}}$ & $|z| < |a|$ \\
		$n ~ a^n ~ u(n)$ & $\ztransform$ & $\dfrac{az^{-1}}{(1 - az^{-1})^2}$ & $|z| > |a|$ \\
		$-n ~ a^n ~ u(-n-1)$ & $\ztransform$ & $\dfrac{az^{-1}}{(1 - az^{-1})^2}$ & $|z| < |a|$ \\
		$d_k(n) ~ a^n ~ u(n)$ & $\ztransform$ & $\dfrac{1}{(1-az^{-1})^k}$ & $|z| > |a|$ \\
		$-d_k(n) ~ a^{n} ~ u(-n-k)$ & $\ztransform$ & $\dfrac{1}{(1-az^{-1})^k}$ & $|z| < |a|$ \\
		$a^{|n|}$ & $\ztransform$ & $\dfrac{1-a^2}{(1-az^{-1})(1-az)}$ & $|a|<|z|<\frac{1}{|a|}$ \\
		$a^n \cos(\omega_{0}n) ~ u(n)$ & $\ztransform$ & $\dfrac{1 - a ~ \cos(\omega_0) ~ z^{-1}}{1-2 a ~ \cos(\omega_{0}) ~ z^{-1} + a^2 ~ z^{-2}}$ & $|z| > |a|$ \\
		$a^n ~ \sin(\omega_0 n) ~ u(n)$ & $\ztransform$ & $\dfrac{a ~ \sin(\omega_{0}) ~ z^{-1}}{1 - 2 a ~ \cos(\omega_{0}) ~ z^{-1} + a^2 ~ z^{-2}}$ & $|z| > |a|$ \\
		$\dfrac{1}{n!} ~ u(n)$ & $\ztransform$ & $e^{z^{-1}}$ & $\mathbb C \setminus \{0\}$
	\end{longtable}
\end{center}

Mit Abkürzung: $d_{1}(n)=1$, $d_{2}(n)=n+1$, $d_{3}(n)=\frac{(n+1)(n+2)}{2}$, d$_{k}(n)=\sum_{i=0}^{n}d_{k-1}(i)$

\subsection*{Rationale $z$-Transformierte}
\vspace{-1.5em}
\begin{multicols}{2}
	\fancyformula{Definition}{
		\[ X(z) = \dfrac{B(z)}{A(z)} = \dfrac{b_0 + b_1 z^{-1} + \cdots + b_M z^{-M}}{a_0 + a_1 z^{-1} + \cdots + a_N z^{-N}} \]
	}

	\fancyformula{Polstellen $p_i$ und Nullstellen $n_i$}{
		\[ X(z = p_i) = \infty ~ \Leftrightarrow ~ B(p_i) = \infty ~ \text{oder} ~ A(p_i) = 0 \]
		\[ X(z = n_i) = 0 ~ \Leftrightarrow ~ B(n_i) = 0 ~ \text{oder} ~ A(n_i) = \infty \]
	}

	\fancyformula{Eigenschaften}{
		\begin{itemize}
			\item Keine Polstellen im KB von $X(z)$
			\item Ein Pol-Nullstellen-Diagramm legt $X(z)$ bis auf eine \textbf{Skalierung} fest
		\end{itemize}
	}

	\fancyformula{Pol-Nullstellen-Zerlegung}{
		\[
			X(z) = \frac{b_{0}}{a_{0}} ~ z^{N-M} ~ \frac{(z-n_{1})\cdot \ldots \cdot(z-n_{M})}{(z-p_{1})\cdot \ldots \cdot(z-p_{N})}
		\]

		\begin{align*}
			\Rightarrow & \begin{cases}
				\text{M nicht triviale Nullstellen } n_{1},...,n_{M}\\
				\text{N nicht triviale Polstellen } p_{1},...,p_{N}
			\end{cases}\\
			+ & \begin{cases}
				N > M\text{: $N - M$ triviale Nullstellen an $z = 0$}\\
				N < M \text{: $M - N$ triviale Polstellen an $z = 0$}
			\end{cases}
		\end{align*}
	}

	{\small\raggedright\sffamily\textbf{Pol-Nullstellen-Kürzung für $p_i = n_j$}}

	\begin{itemize}
		\item Reduzierte Systemordnung
		\item Mögliche Änderung von Stabilität und KB
	\end{itemize}


	% TODO: Zu viel Text, zu lang
	%{\small\raggedright\sffamily\textbf{Zeitverhalten von Polstellen}}
	%
	%Wenn $x(n)$ \textcolor{teal}{rechtsseitig} / \textcolor{red}{linksseitig} ist und wenn:
	%\begin{itemize}
	%	\item Polstellen \textcolor{teal}{innerhalb} / \textcolor{red}{außerhalb} des Einheitskreises liegen, dann klingt $|x(n)|$ ab
	%	\item Polstellen \textcolor{teal}{außerhalb} / \textcolor{red}{innerhalb} des Einheitskreises liegen, dann wächst $|x(n)|$ unbeschränkt
	%	\item Einfache Polstellen auf dem Einheitskreis liegen, dann bleibt $|x(n)|$ beschränkt
	%	\item Mehrfache Polstellen auf dem Einheitskreis liegen, dann wächst $|x(n)|$ unbeschränkt
	%	\item Je kleiner der Abstand von Polstellen zum Einheitskreis ist, desto langsamer das Abklingen/Wachsen von  $|x(n)|$
	%	\item Je größer der Abstand von Polstellen zum Einheitskreis ist, desto schneller das Abklingen/Wachsen von $|x(n)|$
	%\end{itemize}
\end{multicols}

\subsection*{Rücktransformation $X(z) \invztransform x(n)$}
\subsubsection*{Möglichkeit 1: Ablesen mit Potenzreihenenwicklung}
Bringe durch Ausmultiplizieren oder Potenzreihenenwicklung auf Form:
\[
	X(z) \overset{!}{=} \sum_n x(n) ~ z^{-n} ~ \Rightarrow ~ \text{ dann direktes Ablesen von $x(n)$}
\]

\subsubsection*{Möglichkeit 2: Partialbruchzerlegung für rationale $z$-Transformierte}
\[
	\text{Sei} ~ X(z) = \frac{\beta(z^{-1})}{\alpha(z^{-1})} ~ \text{rationale $z$-Transformierte}
\]

\begin{enumerate}
\item \textbf{Polynomdivision} falls Zählergrad $\geq$ Nennergrad (``negativste'' Potenz zuerst)!
\item \textbf{Partialbruchzerlegung} und Bestimmung der Koeffizienten wie herkömmlich oder mit Regeln:
\begin{figure}[H]
	\centering
	\setlength{\fboxsep}{10pt}
	\begin{adjustbox}{valign=h,center=0.42\textwidth,margin=1ex,minipage=[t][][t]{0.37\textwidth},bgcolor=white,cfbox=black 1.5pt 0pt 1pt}
		Koeffizient $c_i$ für \textbf{einfache Polstelle} $p_i$
		\[ \frac{\beta(z^{-1})}{\alpha(z^{-1})} = \frac{c_1}{1 - p_1 z^{-1}} + \cdots + \frac{c_N}{1 - p_N z^{-1}} \]
		\[ c_i = \frac{\beta(z^{-1})}{\alpha(z^{-1})} ~ (1 - p_i z^{-1}) \bigg |_{z = p_{i}} \]
		\vspace{3.8em}
	\end{adjustbox}
	\begin{adjustbox}{valign=h,center=0.48\textwidth,margin=1ex,minipage=[t][][t]{0.43\textwidth},bgcolor=white,cfbox=black 1.5pt 0pt 1pt}
		Koeffizienten $c_1, \ldots, c_m$ für \textbf{m-fache Polstellen} $p_1, \ldots, p_m$
		\[ \frac{\beta(z^{-1})}{\alpha(z^{-1})}=\frac{c_{1}}{1-p_{1}z^{-1}}+\cdots+\frac{c_{m}}{(1-p_{1}z^{-1})^m} + \ldots \]
		\[ c_{m - i} = \frac{1}{i! ~ (-p_1)^{i}}\frac{\mathrm d^{i} \gamma(w)}{\mathrm d w^{i}} \bigg |_{w = p_1^{-1}} \text{mit} ~ 0 \leq i < m \]
		\[ \gamma(w) := \frac{\beta(z^{-1})}{\alpha(z^{-1})} ~ \left(1 - p_1 z^{-1} \right)^m \bigg |_{z^{-1} = w} \]
	\end{adjustbox}
\end{figure}

\item \textbf{Rücktransformation} mittels Tabelle bekannter $z$-Transformationen

\end{enumerate}

\section*{LTI-Systeme / Filter und $z$-Transformation}
\subsection*{Übertragungsfunktion}
\[
	y(n) = (h * x)(n) \ztransform Y(z) = H(z) ~ X(z)
\]
\[
	H(z) = \frac{Y(z)}{X(z)} = \frac{\beta(z^{-1})}{\alpha(z^{-1})}
\]
	Differenzengleichung im Zeitbereich $\leftrightarrow H(z)$ rational mit Pol-und Nullstellen\\

\subsection*{BIBO-Stabilität}
\begin{align*}
	& \text{System ist BIBO-stabil} \\
	\Leftrightarrow \quad & \textstyle \sum_{n=-\infty}^\infty |h(n)| < \infty
	& \Leftrightarrow \quad & \text{$h(n)$ ist kausal und $|p_i| < 1 ~ \forall i$} \\
	\Leftrightarrow \quad & \text{KB von $H(z)$ enthält den Einheitskreis $|z| = 1$}
	& \Leftrightarrow \quad & \text{$h(n)$ ist antikausal und $|p_i| > 1 ~ \forall i$}
\end{align*}

\subsection*{Bezeichnungen FIR und IIR}
\begin{center}
	\begin{tabular}{r c p{8cm}}
		\textbf{Bezeichnung} & \textbf{Übertragungsfunktion} & \textbf{Eigenschaften} \\ \hline \\[-1.0em]
		$\mathrm{FIR}(M)$ & $H(z) = \beta \left(z^{-1} \right)$ & Nichtrekursive Differenzengleichung, endliche Impulsantwort, $\beta \left(z^{-1} \right)$ vom Grad $M$ \\
		$\mathrm{IIR}(N, 0)$ & $H(z) = \frac{1}{\alpha \left(z^{-1} \right)}$ & Rein rekursive Differenzengleichung, nicht-endliche Impulsantwort, $\alpha \left( z^{-1} \right)$ vom Grad $N$ \\
		$\mathrm{IIR}(N, M)$ & $H(z) = \frac{\beta \left(z^{-1} \right)}{\alpha \left(z^{-1} \right)}$ & Allgemein rekursive Differenzengleichung, nicht-endliche Impulsantwort, $\beta \left(z^{-1} \right)$ vom Grad $M$, $\alpha \left( z^{-1} \right)$ vom Grad $N$
	\end{tabular}
\end{center}

\subsection*{Stabilität und Kausalität im Pol-Nullstellen-Diagramm}
\vspace{-1.5em}
\begin{figure}[H]
	\begin{subfigure}[t]{0.33\textwidth}
		{\footnotesize\raggedright\sffamily\textbf{Stabiles System - Einheitskreis im KB}}
		\vspace{0.5em}

		\scalebox{0.8}{
			\begin{pspicture}(-3,-3)(3,3)
				% KB des stabilen Systems
				\psring[fillcolor=gray!20](0, 0){.75}{2.25}

				% Koordinatensystem
				\psaxeslabels{->}(0,0)(-3,-3)(3,3){$\Re$}{$\Im$}

				% Einheitskreis
				\psset{linecolor=black, labelsep=.05}
				\pscircle(0, 0){1.5}
				\rput[b]{45}(1.68; 135){{\scriptsize\color{black} Einheitskreis}}
				\pscircle[style=Dash, linecolor=gray](0, 0){.75}
				\pspole(.75, 0){p1} \nput{-45}{p1}{$\tfrac{1}{2}$}
				\pscircle[style=Dash, linecolor=gray](0, 0){2.25}
				\pspole(-2.25, 0){p2} \nput{225}{p2}{$\scriptstyle-\tfrac{3}{2}$}
			\end{pspicture}
		}
	\end{subfigure}
	\begin{subfigure}[t]{0.33\textwidth}
		{\footnotesize\raggedright\sffamily\textbf{Kausales System - KB ist Äußeres}}
		\vspace{0.5em}

		\scalebox{0.8}{
			\begin{pspicture}(-3,-3)(3,3)
				% KB des kausalen Systems
				\psdiskc[fillcolor=gray!20](0,0)(2.5, 2.5){2.25}

				% Koordinatensystem
				\psaxeslabels{->}(0,0)(-3, -3)(3, 3){$\Re$}{$\Im$}

				% Einheitskreis
				\psset{linecolor=black, labelsep=.05}
				\pscircle(0, 0){1.5}
				\rput[b]{45}(1.68; 135){{\scriptsize\color{black} Einheitskreis}}
				\pscircle[style=Dash, linecolor=gray](0, 0){.75}
				\pspole(.75, 0){p1} \nput{-45}{p1}{$\tfrac{1}{2}$}
				\pscircle[style=Dash, linecolor=gray](0, 0){2.25}
				\pspole(-2.25, 0){p2} \nput{225}{p2}{$\scriptstyle-\tfrac{3}{2}$}
			\end{pspicture}
		}
	\end{subfigure}
	\begin{subfigure}[t]{0.33\textwidth}
		{\footnotesize\raggedright\sffamily\textbf{Antikausales System - KB ist Inneres}}
		\vspace{0.5em}

		\scalebox{0.8}{
			\begin{pspicture}(-3,-3)(3,3)
				% KB des antikausalen Systems
				\psdisk[fillcolor=gray!20](0, 0){0.75}

				% Koordinatensystem
				\psaxeslabels{->}(0,0)(-3, -3)(3, 3){$\Re$}{$\Im$}

				% Einheitskreis
				\psset{linecolor=black, labelsep=.05}
				\pscircle(0, 0){1.5}
				\rput[b]{45}(1.68; 135){{\scriptsize\color{black} Einheitskreis}}
				\pscircle[style=Dash, linecolor=gray](0, 0){.75}
				\pspole(.75, 0){p1} \nput{-45}{p1}{$\tfrac{1}{2}$}
				\pscircle[style=Dash, linecolor=gray](0, 0){2.25}
				\pspole(-2.25, 0){p2} \nput{225}{p2}{$\scriptstyle-\tfrac{3}{2}$}
			\end{pspicture}
		}
	\end{subfigure}
\end{figure}

\subsection*{Definitionen $H \left(e^{j\omega} \right)$, $A(\omega)$, $\theta(\omega)$, $\tau_g(\omega)$}
\vspace{-1.5em}
\begin{multicols}{2}
	\fancyformula{Frequenzgang}{
		\[ H \left(e^{j\omega} \right) = H \left(z = e^{j \omega} \right) \]
	}

	\fancyformula{Amplitudengang}{
		\[ A(\omega) = \left| H \left( e^{j \omega} \right) \right| \]
	}

	\fancyformula{Phasengang}{
		\[ \theta(\omega) = \arg \left( H \left( e^{j \omega} \right) \right) \]
	}

	\fancyformula{Gruppenlaufzeit}{
		\[ \tau_g(\omega) = - \frac{\mathrm d\theta(\omega)}{\mathrm d\omega} \]
	}
\end{multicols}

\subsection*{Kerbfilter}
\vspace{-1.5em}
\begin{figure}[H]
	\centering
	\begin{subfigure}{0.45\textwidth}
		\textbf{FIR}-Kerbfilter hat nur Nullstelle:
		\[
			H(z) = c ~ (1 - 2 \cos(\omega_0) z^{-1} + z^{-2})
		\]

		\textbf{IIR}-Kerbfilter hat Polstelle mit $0 \ll r < 1$:
		\[
			H(z) = c ~ \frac{1 - 2 \cos(\omega_0) z^{-1} + z^{-2}}{1 - 2r ~ \cos(\omega_0) z^{-1} + r^2 ~ z^{-2}}
		\]
	\end{subfigure}
	\begin{subfigure}{0.3\textwidth}
		\scalebox{0.8}{
			\begin{pspicture}(-3,-3)(3,3)
				% KB des kausalen, stabilen Systems
				\psdiskc[fillcolor=gray!20](0,0)(2.7, 2.7){1.838}

				% Koordinatensystem
				\psaxeslabels{->}(0,0)(-3, -3)(3, 3){$\Re$}{$\Im$}

				% Winkel-Linien
				\psline[linestyle=dashed, linecolor=gray](3; 45)
				\psarc[style = Arrow, arrows = <->](0 ,0){1.0}{0}{45}
				\rput(0.7; 22.5){$\omega_0$}

				\psline[linestyle=dashed, linecolor=gray](3; -45)
				\psarc[style = Arrow, arrows = <->](0 ,0){1.0}{-45}{0}
				\rput(0.7; -22.5){$\omega_0$}

				% Einheitskreis
				\psset{linecolor=black, labelsep=.05}
				\pscircle(0, 0){2}

				% Polstellen
				\pspole[linecolor=red](1.3, 1.3){p1} \nput{200}{p1}{\textcolor{red}{$\scriptstyle r e^{j\omega_0}$}}
				\pspole[linecolor=red](1.3, -1.3){p1} \nput{160}{p1}{\textcolor{red}{$\scriptstyle r e^{-j\omega_0}$}}

				% Nullstellen
				\pszero(1.414, 1.414){z1} \nput{20}{z1}{$e^{j \omega_0}$}
				\pszero(1.414, -1.414){z1} \nput{0}{z1}{$e^{-j \omega_0}$}
			\end{pspicture}
		}
	\end{subfigure}
\end{figure}

\subsection*{Kammfilter}
\vspace{-1.5em}
\begin{figure}[H]
	\centering
	\begin{subfigure}{0.45\textwidth}
		Übertragungsfunktion des Kammfilters:
		\[
			H(z) = c ~ \frac{1 - z^{-N}}{1 - r ~ z^{-N}}
		\]
		Polstellen haben Betrag $|p_i| = \sqrt[N]{r}$
	\end{subfigure}
	\begin{subfigure}{0.3\textwidth}
		\scalebox{0.8}{
			\begin{pspicture}(-3,-3)(3,3)
				% KB des kausalen, stabilen Systems
				\psdiskc[fillcolor=gray!20](0,0)(2.7, 2.7){1.838}

				% Koordinatensystem
				\psaxeslabels{->}(0,0)(-3, -3)(3, 3){$\Re$}{$\Im$}

				% Winkel-Linien
				\psline[linestyle=dashed, linecolor=gray](3; 0)
				\psline[linestyle=dashed, linecolor=gray](3; 72)
				\psline[linestyle=dashed, linecolor=gray](3; 144)
				\psline[linestyle=dashed, linecolor=gray](3; 216)
				\psline[linestyle=dashed, linecolor=gray](3; 288)

				% Einheitskreis
				\psset{linecolor=black, labelsep=.05}
				\pscircle(0, 0){2}

				% Polstellen
				\pspole[linecolor=red](1.8, 0){p0}
				\pspole[linecolor=red](0.556, 1.712){p1}
				\pspole[linecolor=red](-1.456, 1.058){p2}
				\pspole[linecolor=red](-1.456, -1.058){p3}
				\pspole[linecolor=red](0.556, -1.712){p4}

				% Nullstellen
				\pszero(2, 0){z0}
				\pszero(0.618, 1.902){z1}
				\pszero(-1.618, 1.176){z2}
				\pszero(-1.618, -1.176){z3}
				\pszero(0.618, -1.902){z4}
			\end{pspicture}
		}
		\[ N = 5 \]
	\end{subfigure}
\end{figure}


\section*{Einseitige $z$-Transformation}
\subsection*{Definition}
\vspace{-1.5em}
\begin{multicols}{2}
	\noindent
	\[ x(n) \ztransformOneSided \sum_{n = \textcolor{red}{0}}^{\infty} x(n) ~ z^{-n} \]
	\[ x(n) = \frac{1}{2\pi j}\oint_C X^{+}(z) ~ z^{n-1} \mathrm dz ~ \text{mit} ~ \textcolor{red}{n \geq 0}\]
\end{multicols}

\subsection*{Eigenschaften}
\vspace{-1.5em}
\begin{multicols}{2}
	\fancyformula{Spiegelung}{
 		\[ x(-n) \ztransformOneSided \text{nicht definiert} \]
	}

	\fancyformula{konjugiert komplex}{
		\[ x^{*}(n) \ztransformOneSided X^{+*}(z^{*}) \]
	}

	\fancyformula{Linearität}{
		\[ \sum_i c_i ~ x_i(n) \ztransformOneSided \sum_i c_{i} ~ X^{+}_i(z) \]
	}

	\fancyformula{Verschiebung mit $k > 0$}{
		\[ x(n - k) \ztransformOneSided z^{-k} \left[X^{+}(z) + \textcolor{red}{\sum_{n = -k}^{-1} x(n) ~ z^{-n}}\right] \]
		\[ x(n + k) \ztransformOneSided z^k \left[X^{+}(z) - \textcolor{red}{\sum_{n = 0}^{k-1} x(n) ~ z^{-n}}\right] \]
	}
	\fancyformula{Dehnung in $z$-Ebene}{
		\[ a^n ~ x(n) \ztransformOneSided X^+\left(\frac{z}{a}\right) ~ \text{mit} ~ a \neq 0\]
	}

	\fancyformula{Differentiation in $z$-Ebene}{
		\[ n ~ x(n)\ztransformOneSided -z ~ \frac{\mathrm d}{\mathrm dz} X^+(z) \]
	}

	\fancyformula{Faltung}{
		\[ (x_1 * x_2)(n) \ztransformOneSided X_1^+(z) ~ X_2^+(z) ~ \text{mit \textcolor{red}{$x_{i}(n)$ rechtsseitig}} \]
	}

	\fancyformula{Moment}{
		\[ \sum_{n = \textcolor{red}{0}}^{\infty} n ~ x(n) = -z ~ \frac{\mathrm d}{\mathrm dz} X^{+}(z) \bigg |_{z = 1} \]
	}

	\fancyformula{Anfangswert}{
	\[x(0)=X(\infty)\]
	}
\end{multicols}
\subsection*{Lösen von Differenzengleichungen}
\begin{enumerate}
	\item Differenzengleichung einseitig $z$-transformieren und Anfangsbedingungen einsetzen
	\[
		y(n) \ztransformOneSided Y^+(z), \quad y(n - 1) \ztransformOneSided z^{-1} ~ Y^+(z) ~ + ~ y(-1), \quad y(n - 2) = z^{-2} ~ Y^+(z) ~ + ~ z^{-1} ~ y(-1) ~ + ~ y(-2),
	\]
	\[
		y(n - 3) = z^{-3} ~ Y^+(z) ~ + ~ z^{-2} ~ y(-1) ~ + ~ z^{-1} ~ y(-2) ~ + ~ y(-3), ~ \text{usw.}
	\]
	\item Einseitige $z$-Transformierte nach $Y^+(z)$ umformen
	\item Durch Rücktransformation $y^+(n)$ bestimmen, z.B. mit Partialbruchzerlegung
	\item  $y(n)$ aus Anfangsbedingungen und $y^+(n)$  bestimmen
\end{enumerate}

\section*{Korrelationsanalyse}
\begin{multicols}{2}
	\begin{adjustbox}{valign=t,center=0.48\textwidth,margin=1ex,minipage=[t][][t]{0.43\textwidth},bgcolor=white,cfbox=black 1.5pt 0pt 1pt}
		\textbf{Korrelation} der Signale $x(n)$, $y(n) \in \mathbb C, ~ 1 \leq n \leq N$:
		\[ r_{xy} = \alpha \sum_{n = 1}^N x(n) ~ y^*(n) \]

		Wähle üblicherweise $\alpha = 1$ oder $\alpha = \frac{1}{N}$
	\end{adjustbox}

	\begin{adjustbox}{valign=t,center=0.48\textwidth,margin=1ex,minipage=[t][][t]{0.43\textwidth},bgcolor=white,cfbox=black 1.5pt 0pt 1pt}
		\textbf{Kovarianz} ($\hat =$ mittelwertbereinigte Korrelation) der Signale $x(n)$, $y(n) \in \mathbb C, ~ 1 \leq n \leq N$:
		\[ c_{xy} = \alpha \sum_{n = 1}^N (x(n) - m_x) ~ (y(n) - m_y)^* \]
		\[ m_x = \frac{1}{N} \sum_{n = 1}^N x(n) \qquad m_y = \frac{1}{N} \sum_{n = 1}^N y(n) \]

		Wähle üblicherweise $\alpha = 1$ oder $\alpha = \frac{1}{N}$
	\end{adjustbox}

	\begin{adjustbox}{valign=t,center=0.48\textwidth,margin=1ex,minipage=[t][][t]{0.43\textwidth},bgcolor=white,cfbox=black 1.5pt 0pt 1pt}
		\textbf{Varianz} des Signals $x(n) \in \mathbb C, ~ 1 \leq n \leq N$:
		\[ c_{xx} = \alpha \sum_{n = 1}^N \left| x(n) - m_x \right|^2 \geq 0 \]

		Wähle üblicherweise $\alpha = 1$ oder $\alpha = \frac{1}{N}$
	\end{adjustbox}

	\begin{adjustbox}{valign=t,center=0.48\textwidth,margin=1ex,minipage=[t][][t]{0.43\textwidth},bgcolor=white,cfbox=black 1.5pt 0pt 1pt}
		\textbf{Korrelationskoeffizient} (normierte Kovarianz) der Signale $x(n), y(n) \in \mathbb C, ~ 1 \leq n \leq N$:
		\[ \rho_{xy} = \frac{c_{xy}}{\sqrt{c_{xx} ~ c_{yy}}} \]

		Wähle üblicherweise $\alpha = 1$ oder $\alpha = \frac{1}{N}$
	\end{adjustbox}
\end{multicols}

% TODO: Kreuzkorrelationsfunktion und Kreuzkoviaranzfunktion
% TODO: Allpass
% TODO: Minimalphasig, Maximalphasig, Linearphasig
% TODO: DFT
% TODO: DFT als Matrix-Vektor-Operation
% TODO: FFT
% TODO: Zirkulare Faltung
% TODO. Eventuell: Schnelle Faltung und Blockfilterungs-Algorithmen?

\section*{Mathematik}
\subsection*{Summenformeln}
\[ \sum_{k = M}^{N - 1} q^k = \frac{q^M - q^N}{1 - q}, ~ q \neq 1  \qquad \qquad \sum_{k = 0}^{\infty} q^k = \frac{1}{1 - q}, ~ |q| < 1 \]
\[ \sum_{k = 1}^{n} k = \frac{n^2 + n}{2} \]

\subsection*{Sinus-Cosinus-Wertetabelle}
\begin{center}
	\renewcommand{\arraystretch}{1.6}
	\footnotesize

	\begin{tabular}{r|c|c|c|c|c|c|c|c|c|c|c|c|c}
		  & 0 & $\frac{\pi}{12}$ & $\frac{\pi}{6}$ & $\frac{\pi}{4}$ & $\frac{\pi}{3}$ & $\frac{5\pi}{12}$ & $\frac{\pi}{2}$ & $\frac{7\pi}{12}$ & $\frac{2\pi}{3}$ & $\frac{3\pi}{4}$ & $\frac{5\pi}{6}$ & $\frac{11\pi}{12}$ & $\pi$ \\ \cline{2-14}	&$0^\circ$&$15^\circ$&$30^\circ$&$45^\circ$&$60^\circ$&$75^\circ$&$90^\circ$&$105^\circ$&$120^\circ$&$135^\circ$&$150^\circ$&$165^\circ$&$180^\circ$\\\hline
		$\sin(x)$&$0$&$\frac{\sqrt{6}-\sqrt{2}}{4}$&$\frac{1}{2}$&$\frac{\sqrt{2}}{2}$&$\frac{\sqrt{3}}{2}$&$\frac{\sqrt{6}+\sqrt{2}}{4}$&1&$\frac{\sqrt{6}+\sqrt{2}}{4}$&$\frac{\sqrt{3}}{2}$&$\frac{\sqrt{2}}{2}$&$\frac{1}{2}$&$\frac{\sqrt{6}-\sqrt{2}}{4}$&$0$\\\hline
		$\cos(x)$&$1$&$\frac{\sqrt{6}+\sqrt{2}}{4}$&$\frac{\sqrt{3}}{2}$&$\frac{\sqrt{2}}{2}$&$\frac{1}{2}$&$\frac{\sqrt{6}-\sqrt{2}}{4}$&0&$-\frac{\sqrt{6}-\sqrt{2}}{4}$&$-\frac{1}{2}$&$-\frac{\sqrt{2}}{2}$&$-\frac{\sqrt{3}}{2}$&$-\frac{\sqrt{6}+\sqrt{2}}{4}$&$-1$\\\hline
	\end{tabular}

	\vspace{0.5em}

	\begin{tabular}{r|c|c|c|c|c|c|c|c|c|c|c|c}
		  & $\frac{13\pi}{12}$ & $\frac{7\pi}{6}$ & $\frac{5\pi}{4}$ & $\frac{4\pi}{3}$ & $\frac{17\pi}{12}$ & $\frac{3\pi}{2}$ & $\frac{19\pi}{12}$ & $\frac{5\pi}{4}$ & $\frac{7\pi}{4}$ & $\frac{11\pi}{6}$ & $\frac{23\pi}{12}$ & $2\pi$ \\ \cline{2-13}	&$195^\circ$&$210^\circ$&$225^\circ$&$240^\circ$&$255^\circ$&$270^\circ$&$285^\circ$&$300^\circ$&$315^\circ$&$330^\circ$&$345^\circ$&$360^\circ$\\\hline
		$\sin(x)$&$-\frac{\sqrt{6}-\sqrt{2}}{4}$&$-\frac{1}{2}$&$-\frac{\sqrt{2}}{2}$&$-\frac{\sqrt{3}}{2}$&$-\frac{\sqrt{6}+\sqrt{2}}{4}$&$-1$&$-\frac{\sqrt{6}+\sqrt{2}}{4}$&$-\frac{\sqrt{3}}{2}$&$-\frac{\sqrt{2}}{2}$&$-\frac{1}{2}$&$-\frac{\sqrt{6}-\sqrt{2}}{4}$&$0$\\\hline
		$\cos(x)$&$-\frac{\sqrt{6}+\sqrt{2}}{4}$&$-\frac{\sqrt{3}}{2}$&$-\frac{\sqrt{2}}{2}$&$-\frac{1}{2}$&$-\frac{\sqrt{6}-\sqrt{2}}{4}$&$0$&$\frac{\sqrt{6}-\sqrt{2}}{4}$&$\frac{1}{2}$&$\frac{\sqrt{2}}{2}$&$\frac{\sqrt{3}}{2}$&$\frac{\sqrt{6}+\sqrt{2}}{4}$&$1$\\\hline	
	\end{tabular}
\end{center}

\end{document}
